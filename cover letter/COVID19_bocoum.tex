%% start of file `template.tex'.
%% Copyright 2006-2013 Xavier Danaux (xdanaux@gmail.com).
%
% This work may be distributed and/or modified under the
% conditions of the LaTeX Project Public License version 1.3c,
% available at http://www.latex-project.org/lppl/.


\documentclass[11pt,a4paper,sans]{moderncv}        % possible options include font size ('10pt', '11pt' and '12pt'), paper size ('a4paper', 'letterpaper', 'a5paper', 'legalpaper', 'executivepaper' and 'landscape') and font family ('sans' and 'roman')

% moderncv themes
\moderncvstyle{casual}                             % style options are 'casual' (default), 'classic', 'oldstyle' and 'banking'
\moderncvcolor{blue}                               % color options 'blue' (default), 'orange', 'green', 'red', 'purple', 'grey' and 'black'
%\renewcommand{\familydefault}{\sfdefault}         % to set the default font; use '\sfdefault' for the default sans serif font, '\rmdefault' for the default roman one, or any tex font name
%\nopagenumbers{}                                  % uncomment to suppress automatic page numbering for CVs longer than one page

% character encoding
\usepackage[utf8]{inputenc}                       % if you are not using xelatex ou lualatex, replace by the encoding you are using
%\usepackage{CJKutf8}                              % if you need to use CJK to typeset your resume in Chinese, Japanese or Korean

% adjust the page margins
\usepackage[scale=0.75]{geometry}
%\setlength{\hintscolumnwidth}{3cm}                % if you want to change the width of the column with the dates
%\setlength{\makecvtitlenamewidth}{10cm}           % for the 'classic' style, if you want to force the width allocated to your name and avoid line breaks. be careful though, the length is normally calculated to avoid any overlap with your personal info; use this at your own typographical risks...

% personal data
\name{Maïmouna}{BOCOUM}
\title{Resumé title}                               % optional, remove / comment the line if not wanted
\address{18 rue du 14 Juillet}{94270}{Kremlin-Bicêtre}% optional, remove / comment the line if not wanted; the "postcode city" and and "country" arguments can be omitted or provided empty
\phone[mobile]{06 14 06 41 50}                   % optional, remove / comment the line if not wanted
\email{maimouna.bocoum@espci.fr}                               % optional, remove / comment the line if not wanted
\extrainfo{permis B}                 % optional, remove / comment the line if not wanted
\photo[64pt][0.4pt]{picture}                       % optional, remove / comment the line if not wanted; '64pt' is the height the picture must be resized to, 0.4pt is the thickness of the frame around it (put it to 0pt for no frame) and 'picture' is the name of the picture file


% to show numerical labels in the bibliography (default is to show no labels); only useful if you make citations in your resume
%\makeatletter
%\renewcommand*{\bibliographyitemlabel}{\@biblabel{\arabic{enumiv}}}
%\makeatother
%\renewcommand*{\bibliographyitemlabel}{[\arabic{enumiv}]}% CONSIDER REPLACING THE ABOVE BY THIS

% bibliography with mutiple entries
%\usepackage{multibib}
%\newcites{book,misc}{{Books},{Others}}
%----------------------------------------------------------------------------------
%            content
%----------------------------------------------------------------------------------
\begin{document}
%-----       letter       ---------------------------------------------------------
% recipient data
\recipient{Maïmouna BOCOUM}{Institut Langevin\\ CNRS UMR7587\\1 rue Jussieu\\Paris 05}
\date{21/03/2020}
\opening{Mme la déléguée régionale,}
\closing{Solidairement,}
\enclosure[Attaché]{curriculum vit\ae{}}          % use an optional argument to use a string other than "Enclosure", or redefine \enclname
\makelettertitle

Suite à l'appel lancé pour la participation à l'élaboration de protocoles de recherche clinique dans le cadre de la crise que traverse actuellement notre pays, je souhaiterais proposer, en toute humilité, ma candidature. En effet, même si mon profil ne me semble pas complètement correspondre aux besoin mentionné dans l'appel, il me semble que mon devoir de chercheuse-citoyenne est d'y répondre, dans le mesure où vous auriez un besoin crucial en "main d’œuvre".\\

Ainsi que l'indique mon CV que vous trouverez joint à la présente, je suis jeune-chercheuse à l'institut Langevin. Ma spécialité est l'optique et je suis une expérimentaliste avant tout. Par ailleurs, j'ai de solides bases en programmation et/ou mathématique. Ainsi, permettez-moi de me positionner en réponse aux différents points mentionnés dans l'appel:

\begin{itemize}
\item  \textbf{Administration de questionnaire téléphonique:} cela ne me semble pas nécessiter de compétence particulière. 
\item  \textbf{Saisie informatique du questionnaire:} cela ne me semble pas nécessiter de compétence particulière. 
\item  \textbf{Aide au data management:} Aucune formation au management, mais très à l'aise en programmation. Prendre en main de nouveaux outils nécessite toutefois une phase d'adaptation.
\item  \textbf{Aide au secrétariat:} Il me semble que du personnel administratif serait plus à même que moi, mais encore une fois, disponible si besoin crucial en main d’œuvre. 
\item \textbf{Manipulation biologique et pilotage d'installations:} Étant physicienne-expérimentaliste, je suis à l'aise avec tout type d'environnement expérimental. Toutefois, je n'ai \textbf{JAMAIS} travaillé sur des projets m'ayant conduit à effectuer des manipulations d'échantillons biologiques et \textbf{AUCUNE} expérience en virologie. Si toutefois des tâches simples avaient besoin d'être effectuées, je suis disponible mais aurais besoin d'être formée aux tâches en question. 
\end{itemize}

J'espère avoir répondu de la manière la plus claire possible à votre appel. J'insiste sur le fait que, ne pensant pas "à priori" correspondre au profil que vous recherchez, je tenais quand même à me manifester dans le cas ou le manque de moyen humain serait tel que vous auriez besoin de moi. Si tel n'était pas le cas, alors d'autres personnes seraient sûrement mieux à même de répondre à votre appel.

\makeletterclosing

\end{document}


%% end of file `template.tex'.
